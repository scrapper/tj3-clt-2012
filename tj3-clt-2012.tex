% This file is Copyright (c) 2012 by Chris Schlaeger <chris@linux.com>
% It's licensed under a NonCommercial-ShareAlike 3.0 Unported (CC BY-NC-SA 3.0)
% license. For details see http://creativecommons.org/licenses/by-nc-sa/3.0/

\documentclass{beamer}

\mode<presentation>
{
  \usetheme{Madrid}
  \setbeamercovered{transparent}
  \setbeamersize{text margin left=1cm}
}

%\usepackage[english]{babel}
\usepackage[latin1]{inputenc}
\usepackage{times}
\usepackage[T1]{fontenc}
\usepackage{listings}
\usepackage{url}

%\title[Projektmanagement mit TaskJuggler]
%{TaskJuggler 3.x - Projektmanagement f�r Linuxanwender}
\title[Projectmanagement with TaskJuggler]
{TaskJuggler 3.x - project management for linux users}

\subtitle
%{Projektmanagement ist mehr als Gantt-Diagramm-Zeichnen}
{project management is more than gantt charts}

\author[C. Schlaeger]{Chris Schlaeger}
\institute{chris@linux.com}
\date[CLT 2012] {Chemitzer Linux-Tage 2012\\{\tiny (english translation by Sebastian Tramp)}}
\subject{project management}
\keywords{TaskJuggler project management gantt linux}

\lstdefinelanguage{tjp}{
  keywords=[1]{now,timezone,timeformat,numberformat,currencyformat,currency,
               extend,text,reference,start,end,effort,length,duration,
       	efficiency,rate,depends,allocate,
       	formats,hidetask,sortasks,hideresource,sortresources,
       	columns, left, header, right, center, navigator, purge},
  keywordstyle={[1]\color{green}},
  keywords=[2]{project,scenario,shift,resource,task,
               textreport,taskreport,resourcereport,tracereport},
  keywordstyle={[2]\color{red}},
  sensitive,
  comment=[s]{/*}{*/},
  morecomment=[l]//,
  commentstyle=\color{grey},
  string=[b]",
  morestring=[b]',
  stringstyle=\color{blue}
}[keywords,comments,strings]

\begin{document}

\lstset{
  language=tjp,
  numbers=left,
  numberstyle=\tiny,
  showstringspaces=false,
  frame=leftline,
  aboveskip=0pt,
  belowskip=0pt
}

\begin{frame}
  \titlepage
\end{frame}

%\begin{frame}{Gliederung}
%  \tableofcontents
%  % Die Option [pausesections] k�nnte n�tzlich sein.
%\end{frame}

\section{Installation and Usage}

\begin{frame}{TaskJuggler and Linux}
  \begin{itemize}
    \item TaskJuggler was initially developed to support the development unit of a Linux distribution
    \item It is still in use by some distributions such as Fedora
    \item It uses a simple project description language
    \item TaskJuggler works similar to a compiler or \LaTeX.
    \item It consists of a few tools which provide the main features and are integrated with other programs
  \end{itemize}
\end{frame}
%\begin{frame}{TaskJuggler und Linux}
  %\begin{itemize}
    %\item TaskJuggler wurde entwickelt um die Entwicklungsabteilung einer Linux-Distribution zu leiten.
    %\item Es wird immer noch von Distributionen wie Fedora verwendet.
    %\item Es verwendet eine einfache Projektbeschreibungssprache.
    %\item TaskJuggler funktioniert �hnlich wie ein Compiler oder \LaTeX.
    %\item Es besteht aus wenigen Programmen, welche die wesentlichen Funktionen liefern und mit anderen Programmen zusammenarbeiten.
  %\end{itemize}
%\end{frame}

\subsection{Installation}

\begin{frame}{Ruby Runtime Environment}
  \begin{itemize}
    \item TaskJuggler 3.x is written in ruby
    \item The main functions are ruby 1.8.7 compatible
    \item The server and scripting functions need at least ruby 1.9.3
    \item We generally suggest to use ruby 1.9.3 since it is 3 times faster than ruby 1.8.7
  \end{itemize}
\end{frame}
%\begin{frame}{Die Ruby Laufzeitumgebung}
  %\begin{itemize}
    %\item TaskJuggler 3.x ist ein Ruby Programm
    %\item Die Grundfunktionen sind noch kompatible zu Ruby 1.8.7.
    %\item Die Server- und Automatisierungsfunktionen erfordern Ruby
    %mindestends 1.9.3.
    %\item Es wird generell Ruby 1.9.3 empfohlen da es etwa 3x schneller
    %ist als 1.8.7.
  %\end{itemize}
%\end{frame}

\begin{frame}[fragile]{Gem Installation}
  \begin{itemize}
    \item Ruby is available on any Linux distribution
    \item Unfortunately many distributions provide only ruby 1.8
    \item RubyGem is the ruby package management tool
    \item ruby gems are OS and CPU architecture independent
    \item The installation mostly easy. A single command downloads the gem package as well as its dependencies and installs everything in you system:
      \begin{verbatim}
gem install taskjuggler \end{verbatim}
  \end{itemize}
\end{frame}
%\begin{frame}[fragile]{Gem Installation}
  %\begin{itemize}
    %\item Alle Linux Distributionen enthalten Ruby.
    %\item Viele Distributionen enthalten leider immer noch das veraltete Ruby 1.8.
    %\item RubyGem ist der Paketmanager f�r Ruby
    %\item Ruby Gems sind vom Prozessor- und vom Betriebssystem unabh�ngig.
    %\item Die Installation ist im Normalfall sehr einfach. Ein
    %Kommando l�dt das Gem Paket und alle Abh�ngigkeiten herunter und
    %installiert diese.
      %\begin{verbatim}
%gem install taskjuggler \end{verbatim}
  %\end{itemize}
%\end{frame}

\begin{frame}[fragile]{Ruby 1.9 Installation}
  \begin{itemize}
    \item The current version is available at \url{ruby-lang.org}
      \url{http://www.ruby-lang.org/en/downloads/}
    \pause
    \item decompress, configure and install
      \begin{verbatim}
tar -Zxvf ruby-X.X.X-*.tar.gz
cd ruby-X.X.X-*
./configure --program-suffix=19
make
sudo make install
ln -s /usr/local/bin/ruby19 ${HOME}/bin/ruby \end{verbatim}
    \pause
    \item \verb|${HOME}/bin| has to be part of your \verb|$PATH| varable
    \pause
    \item Each major ruby version has its own gem package collection
  \end{itemize}
\end{frame}
%\begin{frame}[fragile]{Ruby 1.9 Installieren}
  %\begin{itemize}
    %\item Die aktuelleste Version gibt es auf {\tt ruby-lang.org}
      %\begin{verbatim}
%http://www.ruby-.ang.org/en/downloads/\end{verbatim}
    %\pause
    %\item Auspacken, konfigurieren und installieren
      %\begin{verbatim}
%tar -Zxvf ruby-X.X.X-*.tar.gz
%cd ruby-X.X.X-*
%./configure --program-suffix=19
%make
%sudo make install
%ln -s /usr/local/bin/ruby19 ${HOME}/bin/ruby \end{verbatim}
    %\pause
    %\item \verb|${HOME}/bin| muss im Pfad enthalten sein
    %\pause
    %\item Jede Ruby-Hauptversion hat eine eigene Gem Sammlung
  %\end{itemize}
%\end{frame}

\subsection{Use TaskJuggler}

\begin{frame}[fragile]{Die Online-Hilfe: {\tt tj3man}}
  \begin{itemize}
    \item Get help for a single TaskJuggler keyword
      \begin{verbatim}
tj3man <keyword> \end{verbatim}
    \pause
    \item Mit der {\tt--html} Option wird es im Browser dargestellt
      \begin{verbatim}
tj3man --html <keyword> \end{verbatim}
    \pause
    \item Hilfe zur Hilfe
      \begin{verbatim}
tj3man --help \end{verbatim}
  \end{itemize}
\end{frame}
%\begin{frame}[fragile]{Die Online-Hilfe: {\tt tj3man}}
  %\begin{itemize}
    %\item Hilfe zu einem TaskJuggler Schl�sselwort
      %\begin{verbatim}
%tj3man <keyword> \end{verbatim}
    %\pause
    %\item Mit der {\tt--html} Option wird es im Browser dargestellt
      %\begin{verbatim}
%tj3man --html <keyword> \end{verbatim}
    %\pause
    %\item Hilfe zur Hilfe
      %\begin{verbatim}
%tj3man --help \end{verbatim}
  %\end{itemize}
%\end{frame}

\begin{frame}[fragile]{Das Hauptprogramm: {\tt tj3}}
  \begin{itemize}
    \item Ein Projekt berechnen und Berichte erstellen
      \begin{verbatim}
tj3 yourproject.tjp \end{verbatim}
    \pause
    \item Projekte k�nnen aus mehreren Dateien bestehen
      \begin{verbatim}
tj3 yourproject.tjp reports.tji \end{verbatim}
  \end{itemize}
\end{frame}

\begin{frame}[fragile]{Client und Server: {\tt tj3client} und {\tt tj3d}}
  \begin{itemize}
    \item Den Server starten
      \begin{verbatim}
tj3d -w \end{verbatim}
    \pause
    \item Den Serverstatus abfragen
      \begin{verbatim}
tj3client status \end{verbatim}
    \pause
    \item Ein Projekt (erneut) laden
      \begin{verbatim}
tj3client add yourproject.tjp \end{verbatim}
    \pause
    \item Den Server beenden
      \begin{verbatim}
tj3client terminate \end{verbatim}
  \end{itemize}
\end{frame}

\section{Projekte planen}

\subsection{Die Projektbeschreibungssprache}

\begin{frame}[fragile]{Die Projektbeschreibung}
  \begin{itemize}
    \item Alle Projektdaten werden in einer oder mehreren Textdatein abgelegt.
    \item Information werden als {\it Properties} und zugeh�rigen {\it Attributes} strukturiert.
    \item Eine {\it Property} hat immer folgende Stuktur
      \begin{verbatim}
PROPERTY [ID] "NAME" [ { ATTRIBUTES } ] \end{verbatim}
    \item Die in eckigen Klammern eingeschlossenen Elemente sind optional.
  \end{itemize}
\end{frame}

\begin{frame}[fragile]{Die Properties}
  Ein Projekt besteht aus folgenden {\it Properties}
  \begin{itemize}
    \item {\tt project}: Der Projektkopf
    \pause
    \item {\tt accounts}: Konten zur Kostenplanung
    \pause
    \item {\tt resources}: Mitarbeiter und Arbeitsmittel
    \pause
    \item {\tt tasks}: Die Arbeitsschritte
    \pause
    \item {\tt reports}: Die Berichte
  \end{itemize}
  \pause
  Alle {\it Properties} haben {\it Attributes}. Z. B. die {\it
  Property} {\tt task} hat u. a. die {\it Attributes} {\tt start} und {\tt
  duration}.
\end{frame}

\subsection{Der Projektkopf}

\begin{frame}[fragile]{Der Projektkopf}
  \begin{itemize}
    \item Wie heisst das Projekt?
    \item Wann geht es los?
    \item Wie lange dauert es vorraussichtlich?
  \end{itemize}
  \pause
  \begin{semiverbatim}
    \begin{lstlisting}
project "Beispiel"  2012-03-01 +4m
    \end{lstlisting}
  \end{semiverbatim}
\end{frame}

\begin{frame}[fragile]{Landesspezifische Anpassungen}
  \begin{itemize}
    \item Lokalisierung der Zeitzone und des Datumsformats
    \item Anpassung der Zahlendarstellung
    \item Lokalisierung der Darstellung von Geldbetr�gen
  \end{itemize}
  \pause
  \begin{semiverbatim}
    \begin{lstlisting}
project "Beispiel"  2012-03-01 +4m {
  timezone "Europe/Berlin"
  timeformat "%d.%m.%Y"
  numberformat "-" "" "" "," 1
  currencyformat "-" "" "" "," 0
  currency "EUR"
}
    \end{lstlisting}
  \end{semiverbatim}
\end{frame}

\begin{frame}[fragile]{Mehrere Szenarien}
  \begin{itemize}
    \item Ein Szenario beschreibt eine Variante des Projektverlaufs.
    \item Es k�nnen beliebig viele Szenarien analysiert werden.
    \item Szenarien k�nnen in einer Baumstruktur von einander abgeleitet werden.
    \item Abgeleitete Szenarien erben alle {\it Attributes}, k�nnen sie aber mit eigenen Werten �berschreiben.
  \end{itemize}
  \pause
  \begin{semiverbatim}
    \begin{lstlisting}
project "Beispiel"  2012-03-01 +4m {
  scenario plan "Plan" {
    scenario real "Realit�t"
  }
  now 2012-03-17
}
    \end{lstlisting}
  \end{semiverbatim}
\end{frame}

\begin{frame}[fragile]{Erweiterung des Datenmodells}
  \begin{itemize}
    \item {\it Properties} haben eine Reihe von {\it Attributes}.
    \item Sie k�nnen aber mit eigenen {\it Attributes} erweitert werden.
  \end{itemize}
  \pause
  \begin{semiverbatim}
    \begin{lstlisting}
project example "Beispiel"  2012-03-01 +4m {
  extend resource {
    text Phone "Phone"
  }
  extend task {
    reference Wiki "Wiki"
  }
}
    \end{lstlisting}
  \end{semiverbatim}
\end{frame}

\subsection{Mitarbeiter und Arbeitsmittel}

\begin{frame}[fragile]{Resourcen deklarieren}
  \begin{itemize}
    \item Einen Mitarbeiter anlegen
      \vspace{-15pt}
      \begin{semiverbatim}
        \begin{lstlisting}
resource karl "Karl Mustermann"    
        \end{lstlisting}
      \end{semiverbatim}
    \pause
    \item Ein Team anlegen
      \vspace{-15pt}
      \begin{semiverbatim}
        \begin{lstlisting}
resource ateam "Das A-Team" {
  rate 330.0
  resource karl "Karl Mustermann"
  resource erika "Erika Musterfrau"
}
        \end{lstlisting}
      \end{semiverbatim}
    \pause
    \item Das {\tt rate} {\it Attribute} wird von den verschachtelten
    Resourcen �bernommen.
  \end{itemize}
\end{frame}

\begin{frame}[fragile]{Arbeitsmittel deklarieren}
  \begin{itemize}
    \item Der Ger�t wird nicht m�de, leistet aber auch keine Arbeit.
      \begin{semiverbatim}
        \begin{lstlisting}
resource geraet "Der Ger�t" {
  efficiency 0.0
  rate 500.0
}
        \end{lstlisting}
      \end{semiverbatim}
  \end{itemize}
\end{frame}

\subsection{Die Projektgliederung}

\begin{frame}[fragile]{Die Projektgliederung}
  \begin{semiverbatim}
    \begin{lstlisting}
task "Phase 1" {
  task "Schritt 1"
  task "Schritt 2"
}
task "Phase 2" {
  task "Schritt 1"
  task "Schritt 2"
}
    \end{lstlisting}
  \end{semiverbatim}
\end{frame}

\begin{frame}[fragile]{Abh�ngigkeiten spezifizieren}
  \begin{semiverbatim}
    \begin{lstlisting}
task p1 "Phase 1" {
  task s1 "Schritt 1"
  task "Schritt 2" {
    depends !s1 # Relative ID
  }
}
task p2 "Phase 2" {
  task s1 "Schritt 1" {
    depends p1.s1 # Absolute ID
  }
  task "Schritt 2"
}
    \end{lstlisting}
  \end{semiverbatim}
\end{frame}

\begin{frame}[fragile]{Aufw�nde und Laufzeiten}
  \begin{semiverbatim}
    \begin{lstlisting}
task p1 "Phase 1" {
  task s1 "Schritt 1" {
    duration 2d 
  task "Schritt 2" {
    length 10d
  }
  task "Schritt 3" {
    effort 5d
    allocate karl
  }
}
    \end{lstlisting}
  \end{semiverbatim}
\end{frame}

\subsection{Listen und Berichte}

\begin{frame}{Listen und Berichte}
  \begin{itemize}
    \item Es werden diverse Listenarten unterst�tzt
      \begin{itemize}
        \item Arbeitslisten
	\item Mitarbeiter- und Ger�telisten
	\item Kalender
	\item Zeiterfassungsvorlagen
	\item Statusmeldungsvorlagen
      \end{itemize}
    \pause
    \item Listen k�nnen in unterschiedlichen Formaten erstellt werden
      \begin{itemize}
        \item HTML
	\item CSV
	\item TaskJuggler Syntax
	\item iCal
      \end{itemize}
  \end{itemize}
\end{frame}

\begin{frame}[fragile]{Arbeitslisten}
  \begin{semiverbatim}
    \begin{lstlisting}
taskreport "Arbeitsliste" {
  formats html
  hidetask ~isleaf()
  sorttasks plan.end.up
}
    \end{lstlisting}
  \end{semiverbatim}
\end{frame}

\begin{frame}[fragile]{Mitarbeiterlisten}
  \begin{semiverbatim}
    \begin{lstlisting}
resourcereport "Mitarbeiter" {
  formats html
  sortasks plan.id.up
  columns no, name, email
}
    \end{lstlisting}
  \end{semiverbatim}
\end{frame}

\begin{frame}[fragile]{Arbeitslisten mit Mitarbeitern}
  \begin{semiverbatim}
    \begin{lstlisting}
taskreport "Arbeitsliste" {
  formats html
  hidetask ~isleaf()
  sortasks plan.effort.up
  hideresource 0
  columns no, name, weekly
}
    \end{lstlisting}
  \end{semiverbatim}
\end{frame}

\begin{frame}[fragile]{Textberichte}
  \begin{itemize}
    \item Textberichte bestehen aus bis zu 5 frei definierbaren Textbl�cken.
    \item {\tt header}, {\tt left}, {\tt center}, {\tt right}, {\tt
    bottom}
  \end{itemize}
  \pause
  \begin{semiverbatim}
    \begin{lstlisting}
textreport "Textbericht" {
  formats html
  left "Linker Rand"
  center "Mitte"
  right "Rechter Rand"
}
    \end{lstlisting}
  \end{semiverbatim}
\end{frame}

\begin{frame}[fragile]{{\it RichText} Markup}
  \begin{itemize}
    \item Viele Textattribute werden als {\it RichText} interpretiert.
    \item TaskJuggler verwendet eine Teilmenge des MediaWiki Markups.
    \item Es gibt allerdings einige Erweiterungen:
    \begin{itemize}
      \item Textfarben 
        \begin{verbatim}<fcol:green>Gr�n</fcol>\end{verbatim}
      \item Navigationsleiste
        \begin{verbatim}<[navigator id='mein_menue']>\end{verbatim}
      \item Wertabfrage 
        \begin{verbatim}<-query ...->\end{verbatim}
      \item Eingebetteter Bericht
        \begin{verbatim}<[report id='mein_report']> \end{verbatim}
    \end{itemize}
  \end{itemize}
\end{frame}

\begin{frame}[fragile]{Zusammengesetzte Berichte}
  \begin{semiverbatim}
    \begin{lstlisting}
taskreport r1 "" {
  columns name, chart 
}
resourcereport r2 "" {
  columns name, phone
}
textreport "Textbericht" {
  formats html
  left "<[report id='r1']>"
  right "<[report id='r2']>"
}
    \end{lstlisting}
  \end{semiverbatim}
\end{frame}

\begin{frame}[fragile]{Navigationsleisten f�r HTML Berichte}
  \begin{semiverbatim}
    \begin{lstlisting}
navigator menu
textreport "" {
  header "<[navigator id='menu']>"
  formats html
  taskreport "Aufgaben" {
    columns name, start, end, chart
    hideresource 0
  }
  resourcereport "Mitarbeiter" {
    columns name, email
  }
  purge formats
}
    \end{lstlisting}
  \end{semiverbatim}
\end{frame}

\section{Projekte steuern}

\subsection{Die Projektverfolgung}

\begin{frame}{Der Projektmanagementzyklus}
  Nachdem das Projekt gestartet ist, sollten folgende Schritte
  regelm�ssig durchgef�hrt werden:
  \begin{itemize}
    \item Geleistete Arbeit f�r alle laufenden Aufgaben erfragen
    \item Verbleibenden Aufwand oder Laufzeit erfragen
    \item Plan entsprechend aktualisieren
    \item Vergangenheit einfrieren
  \end{itemize}
\end{frame}

\begin{frame}{{\tt timesheets} und {\tt statussheets}}
  \begin{itemize}
    \item Abfragen der geleisteten und verbleibenden Aufw�nde und
    Laufzeiten kann durch {\tt timesheetreports} stark automatisiert
    werden.
    \item Hierarchische Statusberichte k�nnen mit Hilfe von {\tt
    statussheetreports} automatisiert werden.
    \item Dieses Thema ist zu komplex f�r diesen Vortrag. Es wird im
    Benutzerhandbuch aber ausf�hrlich beschrieben.
  \end{itemize}
\end{frame}

\subsection{Die Vergangenheit einfrieren}

\begin{frame}[fragile]{Die Vergangenheit einfrieren}
  \begin{itemize}
    \item Nach dem Aktualisieren des Plans sollte in regelm�ssigen
    Abst�nden die Vergangenheit eingefroren werden.
    \begin{verbatim}
tj3 --freeze yourproject.tjp \end{verbatim}
    \pause
    \item Es werden 2 Dateien erzeugt oder aktualisiert
      \begin{itemize}
        \item {\tt yourproject-header.tjp}
	\item {\tt yourproject-bookings.tjp}
      \end{itemize}
    \pause
    \item Diese Dateien m�ssen durch {\tt include} in das Projekt
    eingebunden werden.
  \end{itemize}
\end{frame}

\subsection{Kennwerte beobachten}

\begin{frame}{Kennwerte mit {\tt tracereport} beobachten}
  \begin{itemize}
    \item Mit {\tt tracereports} k�nnen wichtige Kennwerte �ber die
    Projektlaufzeit verfolgt werden.
    \item Die Daten werden in eine CSV Datei geschrieben, die bei
    nur bei expliziter Aktualisierung erg�nzt wird.
    \item {\tt tj3 --add-trace yourproject.tjp}
    \item Eignet sich hervorragend zur Erstellung aktueller Burndown
    Charts.
    \item {\tt tracereports} sind die gr��te Neuerung der n�chsten
    Version (3.2).
  \end{itemize}
\end{frame}

\section{Ende}

\begin{frame}{Weiterf�hrende Informationen}
  \begin{itemize}
    \item TaskJuggler im Web: {\tt www.taskjuggler.org}
    \pause
    \item Online Handbuch: {\tt
   www.taskjuggler.org/tj3/manual/index.html}
    \vspace{15pt}
    \pause
    \item Fragen?
  \end{itemize}
\end{frame}

\end{document}


